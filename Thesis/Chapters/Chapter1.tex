% Chapter Template

\chapter{Introduction} % Main chapter title

\label{Chapter1} % Change X to a consecutive number; for referencing this chapter elsewhere, use \ref{ChapterX}

People interact with their environment in unique and nuanced ways, and throughout our lives, humans have learned to identify and categorize the different actions that we perform.

\section{Action Recognition}

For humans, the problem of Human Action Recognition is rather simple. We use past experiences throughout childhood and adult life to be able to pick out the various ways a person moves, and translate that into a familiar action that we have seen before. Combine that with objects that a person may be interacting with, and humans are remarkably good at discerning what actions other humans are involved in. However, as is with most things in the domain of computer vision, this ability does not translate well into the realm of artificial intelligence. The slightly different ways that people may perform these tasks add a layer of complexity that is difficult for a model to overcome.

\section{Applications}

\textbf{Security} is perhaps the most obvious example of action recognition usage. Security personnel are constantly on the lookout for suspicious individuals that may be of concern, or who are performing illegal actions. This can be as simple as trying to find those who are shoplifting in stores, where the CCTV footage can be used live to find those who are actively stealing from stores. It could also be something more complex, such as security checkpoints in airports, where screening officers are constantly watching for suspicious individuals. In this case, a system capable of analyzing the way every person acts and pointing out those who it sees as suspicious could greatly assist those and point them in the correct direction.

\textbf{Health Care} is a slightly different, but nonetheless very interesting application of action recognition. A very large part of how action recognition can help those in the healthcare field is use in monitoring those who need round the clock care, primarily the elderly. If an elderly person chooses to live at home, the action recognition model may allow healthcare workers to, at a distance, manage many people and focus their attention on those who have been flagged as in danger of injury. This can often be done by very lightweight models \cite{Eldermonitoring}.

\textbf{Video Summarization} is perhaps not directly related to action recognition, but rather action recognition is a very useful part of video summarization. If you must summarize a video where the main subjects tend to be humans, a large part of figuring out what is going on in the video is figuring out what action the person is performing, for example, for a summary to be something such as '\textit{The person is fishing.}', the model must have some understanding of what fishing is. Similarly, if the main subject of the video is not a person, it may still be useful to know what those in the background are doing, for example '\textit{A dog is sitting on a bench, there are people doing yoga in the background}', the model again must have an idea of what yoga is, and how a person performs said actions.

\subsection{Ethical Issues}

As with most applications of artificial intelligence, computer vision cannot be researched and discussed without taking into account the ethical issues that surround it. With AI being such a quickly evolving space, it is crucial that any researchers be aware of these issues. In this section, I will focus particularly on how it may affect action recognition, touching on other areas of AI in general to further illustrate my points.

\textbf{Privacy} can become a concern in many areas, the healthcare example given previously in this chapter is one of the most obvious. With elderly people, often one of the draws to staying in their own private homes is the privacy that it offers, if the action recognition model is to be used to ensure that they remain safe, it must come with some removal of privacy. There is also a question of what happens to the data of a person who is using this kind of service, since it is almost certainly sent to a server to be processed given the typical size of these models, what kind of data retention policies might this company have in place, are they sharing this data with others, or is the data going to be used to further train. These are all issues that often follow AI since the training of models requires such a massive amount of data, and in action recognition this can contain people who are not necessarily aware of their data being used in such ways.

%\textbf{Accountability} can be another issue that becomes vague with the application of these AI models. Primarily what 

\textbf{Bias} is one of the most common ethical issues in AI that can appear. In artificial intelligence, and computing in general, the principle of "garbage in, garbage out", is a common one to illustrate that if the inputs into a model are not of high quality, the outputs will not be of high quality. This can often be the case in datasets where say a group of people are not accurately represented, and while the model itself is not discriminatory, it will follow the data it has been given. Take the example previously discussed of airport security. Airport security has been scrutinized in the past for singling out individuals of particular races or who look a particular way. This may mean that if a model is being constructed that searches for people who may be flagged later in security, the majority of positive flags that were screened further would be of this group of people. The resulting dataset that is constructed would be biased against this group of people, therefore resulting in a model similarly biased. This type of issue has been shown in many different areas, another example being speech recognition models used by voice assistants not being able to recognize particular accents as they were not represented in the original dataset. These kinds of reasons are why it is crucial for researchers to be aware of and study their datasets when it comes to human data to avoid these biases and ensure that their data is well balanced.

\textbf{Transparency} is a rather difficult, and often nearly impossible problem to solve with modern AI models. Given the fact that these models at minimum have millions of parameters that all contribute to the complex calculations towards the output, deciphering exactly how they work and make their decisions is difficult. These models are often depicted as black boxes, where the only context we are allowed is what inputs and outputs, and nothing in between. In the cases of something such as an airport security checkpoint, the model may mark a person as acting suspicious in a line of passengers. The model is not able to specifically express what made the passenger appear suspicious, and it may even be incorrect in it's assumptions. This means that the officer who is reviewing the flags set by the model has to make a decision that leads to one of two possible scenarios:

\begin{enumerate}
	\item The model is correct, but cannot communicate it's exact reasoning with the officer, the officer does not see what the model sees and a potential threat is ignored
	\item The model is incorrect, but the officer thinks that he sees something, and a person who is not a threat is put through unnecessary screening, and other potential threats may not be screened
\end{enumerate}

\section{Challenges}
\label{sec:challenges}

Human action recognition is a very difficult task that comes with many issues, some of which continue to be major challenges moving forward with very complex modern models.

\textbf{Backgrounds} often not static in videos. Often they contain a lot of data that is ever changing and often can contain other secondary subjects performing actions that may not be relevant to the subject that we are trying to determine the action of. While humans are very good at focusing on the person who is performing the action and ignoring things occurring in the background enough to not get confused. AI models do not have this inherent ability and often can get confused from background changes, and effectively must both identify the person and determine what action they are performing within the same model. This can be further worsened by any camera movement, resulting in both the subject moving throughout the frame, but the background can completely change with a  90 degree camera angle change

\section{Problem Definition}

The problem of human action recognition is defined as taking a video of a person performing a particular action, and passing it through a model to determine the specified action the person is performing. Pose-based action recognition is the problem of performing this detection primarily using the skeleton data of the people in the frame.

\section{Thesis Structure}

This thesis will begin by exploring the research relevant to action recognition in chapter \ref{LiteratureReview}. Next, a new novel representation is proposed in chapter \ref{Methodology}, and the experiments and their results are detailed in chapter \ref{Experimentation}. Finally the conclusions and recommendations for future work are presented in chapter \ref{Conclusion}.