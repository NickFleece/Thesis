% Chapter Template

\chapter{Experimentation} % Main chapter title

\label{Experimentation} % Change X to a consecutive number; for referencing this chapter elsewhere, use \ref{ChapterX}

\section{Model Hyperparameters}

\subsection{Compute Resources}

\section{JHMDB Results}

\begin{table}[h]
	\centering
	\begin{tabular}{||c c||} 
		\hline
		\textbf{Split} & \textbf{Accuracy} \\ [0.5ex] 
		\hline\hline
		1 & 58.209\% \\ 
		\hline
		2 & 58.889\% \\
		\hline
		3 & 58.113\% \\
		\hline
		\hline
		\textbf{Average} & \textbf{58.404\%} \\
		\hline
	\end{tabular}
	\label{tab:acc-results}
	\caption{Results on all 3 splits of the JHMDB dataset utilizing only our novel approach.}
\end{table}

\begin{figure}[h]
	\includegraphics[width=10cm]{detailedPrecision}
	\centering
	\caption{Detailed precision results on the JHMDB dataset, averaged over the 3 testing splits. Black bars show the corresponding maximum \& minimum values attained in one of the splits.}
	\label{fig:detailed-precision}
\end{figure}

\begin{figure}[h]
	\includegraphics[width=10cm]{detailedRecall}
	\centering
	\caption{Detailed recall results on the JHMDB dataset, averaged over the 3 testing splits. Black bars show the corresponding maximum \& minimum values attained in one of the splits.}
	\label{fig:detailed-recall}
\end{figure}

\begin{figure}[h]
	\includegraphics[width=10cm]{detailedF1}
	\centering
	\caption{Detailed F1 score results on the JHMDB dataset, averaged over the 3 testing splits. Black bars show the corresponding maximum \& minimum values attained in one of the splits.}
	\label{fig:detailed-f1}
\end{figure}

\section{Model Ablation Study}

\begin{table}[h]
	\centering
	\begin{tabular}{||c c c||} 
		\hline
		\textbf{Split} & \textbf{Stacked} & \textbf{Only Angle Velocity} \\ [0.5ex] 
		\hline\hline
		1 & \textbf{58.209\%} & 48.888\% \\ 
		\hline
		2 & \textbf{58.889\%} & 44.444\% \\
		\hline
		3 & \textbf{58.113\%} & 41.132\% \\
		\hline
		\hline
		\textbf{Average} & \textbf{58.404\%} & 44.821\% \\
		\hline
	\end{tabular}
	\label{tab:acc-results-v-velocity}
	\caption{Comparison of results using only angle velocities vs the stacked representation.}
\end{table}

\begin{table}[h]
	\centering
	\begin{tabular}{||c c c||} 
		\hline
		\textbf{Split} & \textbf{Stacked} & \textbf{Only Angles} \\ [0.5ex] 
		\hline\hline
		1 & \textbf{58.209\%} & 56.343\% \\ 
		\hline
		2 & \textbf{58.889\%} & \textbf{58.889\%} \\
		\hline
		3 & \textbf{58.113\%} & 56.604\% \\
		\hline
		\hline
		\textbf{Average} & \textbf{58.404\%} & 57.279\% \\
		\hline
	\end{tabular}
	\label{tab:acc-results-v-angle}
	\caption{Comparison of results using only angles vs the stacked representation.}
\end{table}

\section{Model Comparison}