% Chapter Template

\chapter{Conclusion \& Future Work} % Main chapter title

\label{Conclusion}

In this thesis, we have provided a novel representation for use in action recognition. This novel representation is unique to many other models in that it is invariant to:

\begin{itemize}
	\item Scale
	\item Global Position
	\item 2-D Rotation
\end{itemize}

We explore many different variations of this representation, concluding that the best representation contains both joint angle data and joint angle velocity data, however, only angle data also provides good results. It was also found that this data functions the best as a single image, with one data source stacked on top of another. This means that the combining of the angles and velocities can be learned by the convolutions slowly over the entire model, rather than typical fusion-like architectures that concatenate the different sources at the end.

With this representation, we were able to leverage a simple CNN model that is trainable on lightweight GPU's and is able to be tested on even lighter hardware. Despite the model being relatively lightweight, the model is able to provide 58.404\% average accuracy over the 3 splits of the JHMDB dataset. This shows that despite the model not having been fed any data about how the person is moving globally, any data regarding the objects the person is interacting with, or any data about the environment it is able to fairly accurately predict what action the person is performing.

After further analyzing the results, the accuracy was shown to be slightly more nuanced than simply 58\% accurate overall. The model was able to predict classes that had more consistent movements that varied less from person to person such as golf or pullups, and tended to struggle with movements that can vary from person to person much more such as jump or catch.

\subsection{Future Work}

There is significant room for future work in this specific domain. Utilizing the skeleton data of a person to assist with action recognition is constantly evolving, and has continually been proved to help existing models improve on their results. However, a perhaps more interesting area of research that is not quite as popular is intermediate representations, which is where the "low hanging fruit" are. The first place  of improvement would be to make the representation truly rotationally invariant, this would have to be done through 3-dimensional pose detection, and modifications of how changes in rotations are represented in the end image. 

Another path forward is to more fully examine how real-time pose detection algorithms can be integrated with this simple model in order to obtain very fast and real-time results. This real-time computation is becoming more and more relevant in the world of smartphones, and examining how the model is able to be used on smartphones is crucial to determine the industrial applications. It is also worth noting that implementing this representation into existing complex models may be of interest in some applications, however the appeal of the representation being very lightweight and mobile applications not being as relevant cannot be ignored.