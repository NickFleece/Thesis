% Chapter Template

\chapter{Literature Review} % Main chapter title

\label{Chapter3} % Change X to a consecutive number; for referencing this chapter elsewhere, use \ref{ChapterX}

%Typical examples of action recognition \cite{i3d} have used complex convolutional neural networks (CNNs) on the RGB frames, pulling features such as rgb flow \cite{rgbflow} in order to enhance results. This method has proved to work well, however these models often require large amounts of GPU memory and are required to be run on high end hardware, which is a potential issue when applied to real-world scenarios where the necessary hardware may not be available. In addition, background noise is much more difficult to filter out and generalizing to different environments is difficult.

%A subset of action recognition models utilize skeleton-based action recognition. These models aim to utilize the positions of joints/bones in the model in order to filter out background data, and allow the model to focus on the actual person rather than background data. Typically these involve simpler CNNs that allow for faster computation on lower quality hardware. Many different approaches are used to achieve this action recognition. \cite{potion}, \cite{PA3D} utilize processed joint heatmap images and simple 2D-CNNs. \cite{simple_yet_efficient} \cite{smaller_faster_better} use intermediate representations to construct custom image representations that can be easily processed by simple CNNs. RNNs and LSTMs \cite{two_branch_stacked_lstm} \cite{RNN_joint_relative_motion} \cite{RNN_occlusion} \cite{DS_lstm} and Transformers \cite{transformersnippets} \cite{transformertwobranch} have been used as well to obtain good results with skeleton data.

%Almost always, these skeleton based models utilize 2D pose data. There have been some examples of extracting 3D pose from depth cameras \cite{depthcamera3dpose} as well as estimating 3D pose from 2D pose \cite{3dposefrom2d} and only the RGB frames \cite{2dposefromrgb}. These techniques have been used in the past for human action recognition \cite{3dposeactionrecognition}, however 2D pose estimation is a much easier task and the models that have been used are generally much higher quality, and therefore is more reliable for the task of human action recognition.

%----------------------------------------------------------------------------------------
%	Image Classification
%----------------------------------------------------------------------------------------
\section{Image Classification}

%----------------------------------------------------------------------------------------
%	RGB Models
%----------------------------------------------------------------------------------------
\section{RGB Frame Based Models}

\subsection{CNN + LSTM Models}

\subsection{3D CNN Models}

%----------------------------------------------------------------------------------------
%	Optical Flow
%----------------------------------------------------------------------------------------
\section{Optical Flow}

\subsection{Optical Flow Models}

%----------------------------------------------------------------------------------------
%	Attention
%----------------------------------------------------------------------------------------
\section{Attention Based Models}

%----------------------------------------------------------------------------------------
%	Datasets
%----------------------------------------------------------------------------------------
\section{Datasets}

%----------------------------------------------------------------------------------------
%	Skeleton Based Models
%----------------------------------------------------------------------------------------
\section{Skeleton-Based Action Recognition}

\subsection{Intermediate Representations}